\documentclass[10pt]{article}

\usepackage{blindtext} % Package to generate dummy text throughout this template 

\usepackage[sc]{mathpazo} % Use the Palatino font
\usepackage[T1]{fontenc} % Use 8-bit encoding that has 256 glyphs
\linespread{1.05} % Line spacing - Palatino needs more space between lines
\usepackage{microtype} % Slightly tweak font spacing for aesthetics

\usepackage[english]{babel} % Language hyphenation and typographical rules

\usepackage[hmarginratio=1:1,top=32mm,columnsep=20pt]{geometry} % Document margins
%\usepackage[hang, small,labelfont=bf,up,textfont=it,up]{caption} % Custom captions under/above floats in tables or figures
\usepackage[small,labelfont=bf,up,textfont=it,up]{caption}
\usepackage{sidecap}
\usepackage{floatrow}

\usepackage{booktabs} % Horizontal rules in tables

\usepackage{enumitem} % Customized lists
\setlist[itemize]{noitemsep} % Make itemize lists more compact

\usepackage{abstract} % Allows abstract customization
\renewcommand{\abstractnamefont}{\normalfont\bfseries} % Set the "Abstract" text to bold
\renewcommand{\abstracttextfont}{\normalfont\small\itshape} % Set the abstract itself to small italic text

\usepackage{titlesec} % Allows customization of titles
\renewcommand\thesection{\Roman{section}} % Roman numerals for the sections
\renewcommand\thesubsection{\roman{subsection}} % roman numerals for subsections
\titleformat{\section}[block]{\large\scshape\centering}{\thesection.}{1em}{} % Change the look of the section titles
\titleformat{\subsection}[block]{\large}{\thesubsection.}{1em}{} % Change the look of the section titles

\usepackage{fancyhdr} % Headers and footers
\pagestyle{fancy} % All pages have headers and footers
\fancyhead{} % Blank out the default header
\fancyfoot{} % Blank out the default footer
%\fancyhead[C]{CRes User Manual $\bullet$ 2019 $\bullet$ Watson} % Custom header text
\fancyhead[C]{AirGun1D User Guide~~~$\bullet$~~~Watson~~~$\bullet$~~~2020} % Custom header text
\fancyfoot[RO,LE]{\thepage} % Custom footer text

\usepackage{titling} % Customizing the title section

\usepackage{hyperref} % For hyperlinks in the PDF

% math packages
\usepackage{amssymb}
\usepackage{amsmath}
\usepackage{amsfonts}

% figure packages
\usepackage{graphicx}

\usepackage{natbib} % bibliography

\usepackage{listings} % for including code snippets
\usepackage{color}

\definecolor{codegreen}{rgb}{0,0.6,0}
\definecolor{codegray}{rgb}{0.5,0.5,0.5}
\definecolor{codepurple}{rgb}{0.58,0,0.82}
\definecolor{backcolour}{rgb}{0.95,0.95,0.92}
 
\lstdefinestyle{mystyle}{
    backgroundcolor=\color{backcolour},   
    commentstyle=\color{codegreen},
    keywordstyle=\color{magenta},
    numberstyle=\tiny\color{codegray},
    stringstyle=\color{codepurple},
    basicstyle=\footnotesize,
    breakatwhitespace=false,         
    breaklines=true,                 
    captionpos=b,                    
    keepspaces=true,                 
    %numbers=left,                    
    numbersep=5pt,                  
    showspaces=false,                
    showstringspaces=false,
    showtabs=false,                  
    tabsize=2
}
 
\lstset{style=mystyle}



%----------------------------------------------------------------------------------------
%	TITLE SECTION
%----------------------------------------------------------------------------------------

\setlength{\droptitle}{-4\baselineskip} % Move the title up

\pretitle{\begin{center}\Huge\bfseries} % Article title formatting
\posttitle{\end{center}} % Article title closing formatting
\title{AirGun1D User Guide} % Article title
\author{%
\textsc{Leighton M. Watson} \\% Your name
\normalsize Stanford University \\ % Your institution
\normalsize \href{mailto:leightonwatson@stanford.edu}{leightonwatson@stanford.edu} % Your email address
%\and % Uncomment if 2 authors are required, duplicate these 4 lines if more
%\textsc{Jane Smith}\thanks{Corresponding author} \\[1ex] % Second author's name
%\normalsize University of Utah \\ % Second author's institution
%\normalsize \href{mailto:jane@smith.com}{jane@smith.com} % Second author's email address
}
\date{\today} % Leave empty to omit a date
\renewcommand{\maketitlehookd}{%
}

%----------------------------------------------------------------------------------------

\begin{document}

% Print the title
\maketitle

%----------------------------------------------------------------------------------------
%	ARTICLE CONTENTS
%----------------------------------------------------------------------------------------

\section{Introduction}
{\bf AirGun1D} simulates a one-dimensional airgun coupled to an air bubble surrounded by water. The oscillations of the air bubble in the water generate acoustic waves that are used in marine seismic surveys. the air bubble is described with a lumped parameter model where the internal properties of the bubble are assumed to be spatially uniform. The airgun (or source) is described as one-dimensional such that properties can vary with space. {\bf AirGun1D} is written in \texttt{MATLAB} and runs efficiently on a standard desktop/laptop computer. For more details and examples of the application of {\bf AirGun1D} see:
\begin{itemize}
\item Watson, L. M., Werpers, J., and Dunham, E. M. (2019) What controls the initial peak of an air gun source signature? \emph{Geophysics}, 84 (2), P27-P45, \href{https://doi.org/10.1190/geo2018-0298.1}{https://doi.org/10.1190/geo2018-0298.1}.
\end{itemize}

{\bf AirGun1D} is freely available online at \href{https://github.com/leighton-watson/AirGun1D}{https://github.com/leighton-watson/AirGun1D} and is distributed under the MIT license (see \texttt{license.txt} for details). 

This code extends {\bf SeismicAirgun} (\href{https://github.com/leighton-watson/SeismicAirgun}{https://github.com/leighton-watson/SeismicAirgun}), which describes the airgun (source) with a lumped parameter model. {\bf AirGun1D} is more suitable for large sources where the spatial variations of properties inside the source is important. 

\section{Directory}
\begin{itemize}
\item{\bf SBPSAT} - source code for AirGun1D. Governing equations are solved using the summation-by-parts technique with simultaneous approximation term (SBP-SAT). This directory also contains several folders that have modified versions of the source code that are used in generating the figures of \citet{Watson2019_airgun}.
\item {\bf SeismicAirgunCode} - source code for lumped parameter model description of airgun. This directory also contains functions that are required to calculate acoustic radiation. 
\item {\bf doc} - documentation including user guide and license file.
\item {\bf MakeFigs} - script files that generate all of the figures of \citet{Watson2019_airgun}.
\item {\bf Data} - data files that are needed to when generating the figures of of \citet{Watson2019_airgun}.
\item {\bf BubbleCode} - bubble models that are compared in the appendix of \citet{Watson2019_airgun}.
\end{itemize}

\section{SBPLib}
AirGun1D is a finite-difference code that uses summation-by-parts (SBP) operators \cite{Svard2014} and relies on SBPLib, which is a library of primitives and help functions for working with summation-by-parts finite differences in Matlab developed by Ken Mattsson's group at Uppsala University (including Jonatan Werpers, Martin Almquist, Ylva Rydin and Vidar Stiernstr\"{o}m). SBPLib is hosted through \href{https:bitbucket.org}{bitbucket} and is available at \href{https://bitbucket.org/sbpteam/sbplib/src/default/}{https://bitbucket.org/sbpteam/sbplib/src/default/}. Clone or download SBPLib and add it to your Matlab path:
\begin{lstlisting}[language=Matlab]
addpath ../sbplib/ % add path to SBPLib
\end{lstlisting}

\section{Model Description}
Here, I describe the mathematical model of {\bf AirGun1D}. The model can be divided into three components; i.) airgun, ii.) bubble, and iii.) acoustic radiation. The models for components ii) and iii) are the same as for {\bf SeismicAirgun}. For more details see the {\bf SeismicAirgun} user guide and \citet{Chelminski2019}, \citet{Watson2016} and the references therein. Here, I provide a detailed discussion of the airgun model and a cursory description of the bubble and acoustic radiation models. For more details see \citet{Watson2019_airgun}.

\begin{figure}[h!]
\centering
\includegraphics[width=0.8\textwidth]{Fig_Schematic}
\caption{Schematic diagram of AirGun1D model showing the three model components: i) airgun that is described by the Euler equation that allow for spatial variations of properties within the airgun, ii) bubble that is described with a lumped parameter model and the dynamics governing by the modified Herring equation, and iii) acoustic radiation that is modeled as a point source radiating in a half space.}
\label{fig:schematic}
\end{figure}

\subsection{Airgun}
The air gun firing chamber is idealized as a cylinder of length $L$ and cross-sectional area $A$. The air inside is compressible and inviscid. We only consider waves and flow along the length of the firing chamber, which are governed by the one-dimensional Euler equations:
\begin{align}
\frac{\partial \rho}{\partial t} + \frac{\partial (\rho v)}{\partial x} & = 0, \\
\frac{\partial (\rho v)}{\partial t} + \frac{\partial (\rho v^2 + p)}{\partial x}
& = 0, \\
\frac{\partial e}{\partial t} + \frac{\partial [(e+p)v]}{\partial x} & = 0,
\end{align}
where $\rho$, $p$, and $v$ are the density, pressure, and axial velocity of the fluid inside the air gun, respectively, and $e$ is the internal energy per unit volume. The system is closed with the ideal gas equation of state, written here as
\begin{equation}
p = (\gamma-1)\bigg(e-\frac{1}{2} \rho v^2 \bigg),
\end{equation}
and the internal energy per unit volume is
\begin{equation}
e = c_v \rho T.
\end{equation}

The model is initialized with spatially uniform pressure, density and temperature through the airgun and zero initial velocity. The initial air gun pressure is set equal to the operating pressure, $p_0$, and the initial temperature is assumed to have equilibrated with the ambient temperature $T_{\infty}$ (since the air travels a long distance in a submerged umbilical hose from the compressor on the seismic vessel to the air gun). 
\begin{align}
p_a(x, 0) & = p_0, \\
T_a(x, 0) & = T_{\infty}, \\
v_a(x,0) & = 0.
\end{align}

Boundary conditions are also required for the 1D airgun model. At the back wall, $x=0$, a solid wall boundary condition, $v(0)=0$, is imposed. At the air gun port, $x=L$, the boundary condition depends upon the flow state. If flow out of the port is subsonic, $v(L)<c(L)$, continuity of pressure is required and the pressure in the air gun at the port, $p(L)$, is set equal to the pressure in the bubble, $p_b$. If the port is choked, $v(L)=c(L)$, waves cannot propagate upstream from the bubble into the air gun and pressure continuity is not required. The bubble pressure will be different than the air gun pressure. The latter is generally the case in our simulations.

The source stops discharging (the port closes) as a specified time. This value is set in \texttt{SBPSAT/configAirgun.m}:
\begin{lstlisting}[language=Matlab]
physConst.AirgunCutoffTime = 0.04; % time when air gun stops firing [s]
\end{lstlisting}


\subsection{Bubble}
The bubble is idealized as spherically symmetric with spatially uniform internal properties. The bubble wall dynamics are governed by the modified Herring equation (\citealp{Herring1941,Cole1948,Vokurka1986}; Appendix A):
\begin{equation}
R \ddot{R}+ \frac{3}{2} \dot{R}^2 = \frac{p_b - p_\infty}{\rho_\infty} + \frac{R}{\rho_\infty c_\infty} \dot{p_b} - \alpha \dot{R},
\label{eq:modified herring}
\end{equation}
where $R$ and $\dot{R}=dR/dt$ are the radius and velocity of the bubble wall, respectively, and $p_\infty, \rho_\infty$ and $c_\infty$ are the ambient pressure, density, and speed of sound, respectively, in the water infinitely far from the bubble. The ambient pressure is calculated by adding the hydrostatic pressure to the atmospheric pressure, $p_\infty~=~p_\text{atm}~+~\rho_\infty g D$, where $p_\text{atm}$ is the atmospheric pressure, $g$ is the acceleration from gravity and $D$ is the depth of the air gun below the water. Numerical models of air gun signatures typically under predict the damping observed in the data. This is likely due to the models neglecting various dissipation mechanisms such as turbulence and energy loss to phase changes. Mechanical energy dissipation mechanisms, which are challenging to model directly, are parameterized into an extra damping term of the form of $-\alpha \dot{R}$ where $\alpha$ is an empirically determined constant \citep{Langhammer1996, Watson2017_airgun}. This term dampens the bubble oscillations so that they are in agreement with the observations but has minimal impact on the initial bubble growth. The parameter $\alpha$ is set in \texttt{SBPSAT/bubbleRHS.m}:
\begin{lstlisting}[language=Matlab]
alpha=0.8; % damping parameter 
\end{lstlisting}

The pressure in the bubble is calculated from the ideal gas equation of state,
\begin{equation}
p_b = \frac{m_b Q T_b}{V_b},
\label{eq:bubble eos}
\end{equation}
where $m_b$ is the mass inside the bubble, $T_b$ is the temperature inside the bubble in Kelvin, $V_b$ is the volume of the spherical bubble and the internal energy is
\begin{equation}
E_b = c_v m_b T_b.
\label{eq:bubble int energy}
\end{equation}

The internal energy of the bubble changes due to the advection of mass and associated transport of enthalpy into the bubble, work done by volume changes, and heat loss to the water:
\begin{equation}
\frac{dE_b}{dt} = c_p T_a \frac{dm_b}{dt} - 4\pi M \kappa R^2 (T_b-T_\infty) - p_b \frac{dV_b}{dt},
\label{eq:temp}
\end{equation}
where $T_\infty$ is the ambient temperature in the water, $\kappa$ is the heat transfer coefficient \citep{Ni2011, deGraaf2014} and $M$ is a dimensionless, empirically determined constant that accounts for the increased effective surface area over which heat transfer can occur as a result of turbulence in the bubble dynamics \citep{Laws1990}. Equation~\ref{eq:temp} is a statement of the first law of thermodynamics for an open system \citep{Tolhoek1952}.

Mass flow into the bubble is determined by conservation of mass of the gas exiting the airgun:
\begin{equation}
\frac{dm_b}{dt} = v(L) \rho(L) A.
\label{eq:mass flow}
\end{equation}
The cross-sectional area of the port, $A$, is assumed to be the same as the cross-sectional area of the firing chamber (Figure~\ref{fig:schematic}).

Note that unlike in {\bf SeismicAirgun}, the bubble does not ascend buoyantly in {\bf AirGun1D}. This is equivalent to setting $\beta=0$ where $\beta$ is the tuning parameter that controls the rate of ascent in {\bf SeismicAirgun}.

\subsection{Acoustic Radiation}
The pressure perturbation in the water is related to the bubble dynamics by \citep{Keller1956}
\begin{equation}
\Delta p(r,t) = \frac{\rho_\infty}{4\pi r} {\ddot{V}(t-r/c_\infty)},
\label{eq:acoustic radiation}
\end{equation}
where $\Delta p$ is the pressure perturbation in the water, $r$ is the distance from the center of the bubble to the receiver, and $V$ is the volume of the bubble. The second term on the right side is a near-field term that decays rapidly with distance. In addition to the direct arrival there is a ghost arrival, which is the initially upgoing wave that is reflected with negative polarity from the sea surface. For a receiver directly below the source the observed pressure perturbation is
\begin{equation}
\Delta p_\text{obs}(r,t) = \Delta p_D(r,t) - \Delta p_G(r+2D,t),
\end{equation}
where $\Delta p_D$ is the direct signal and $\Delta p_G$ is the ghost arrival, which is assumed to have -1 reflection coefficient from the sea surface. $D$ is the depth of the seismic airgun.

\section{Numerical Implementation}
The Euler equations inside the air gun are solved using finite difference summation-by-parts operators for spatial derivatives \citep{Svard2014}. The air gun and bubble governing equations are evolved in time using an explicit adaptive Runge-Kutta (4,5) method. The air gun and bubble models are coupled at the interface using the simultaneous-approximation-term framework \citep{Carpenter1994,DelReyFernandez2014a,Svard2014}.
The reader is referred to \citet{Watson2019_airgun} for more details.

\section{Using {\bf AirGun1D}}
A script file, \texttt{example.m}, provides an example on how to set up and run a simulation as well as how to plot the outputs. It is important to add the path to the required directories
\begin{lstlisting}[language=Matlab]
addpath SBPSAT/
addpath SeismicAirgunCode
% addpath sbplib/ % add path to sbplib
\end{lstlisting}

The example script file defines the discretization (number of grid points per one meter length of source) and the source properties (length, area (port area is equal to the cross-sectional area), depth, and pressure):
\begin{lstlisting}[language=Matlab]
% define discretization
M = 50; % number of grid points per 1m of source length

% source  properties
src_length = 1.2; % source length (m)
src_area = 12.5; % cross-sectional area of source (in^2)
src_depth = 7.5; % source depth (m)
src_pressure = 1000; % source pressure (psi)
\end{lstlisting}

The solver is run by calling:
\begin{lstlisting}[language=Matlab]
sol = runEulerCode(nx, src_pressure, src_length, src_area, src_depth);
\end{lstlisting}

\begin{figure}[b!]
\centering
\includegraphics[width=\textwidth]{Fig_soly}
\caption{Format of outputs from \texttt{runEulerCode}. Value in parenthesis indicates the time step e.g., $1, 2, \hdots, N$ while subscript indicates spatial position for the variables in the source e.g., $1, \hdots, M$.}
\label{fig:sol y}
\end{figure}

%% sol.y equation format - for illustrator file
%\begin{equation}
%\texttt{sol.y} = 
%\begin{bmatrix}
%R(1) & R(2) & \hdots & R(N) \\
%U(1) & U(2) & \hdots & U(N) \\
%m_b(1) & m_b(2) & \hdots & m_b(N) \\
%E_b(1) & E_b(2) & \hdots & E_b(N) \\
%\rho_0(1) & \rho_0(2) & \hdots & \rho_0(N) \\
%\rho_0(1) v_0(1) & \rho_0(2) v_0(2)& \hdots & \rho_0(N) v_0(N)\\
%e_0(1) & e_0(2) & \hdots & e_0(N) \\
%\vdots & \vdots & & \vdots \\
%\rho_M(1) & \rho_M(2) & \hdots & \rho_M(N) \\
%\rho_M(1) v_M(1) & \rho_M(2) v_M(2)& \hdots & \rho_M(N) v_M(N)\\
%e_M(1) & e_M(2) & \hdots & e_M(N) \\
%\end{bmatrix}
%\end{equation}

The time vector is saved in \texttt{sol.x} and has length $N$, where $N$ is the number of time steps. The bubble and airgun variables are saved in \texttt{sol.y}. The bubble variables that are solved for are the bubble radius, $R$, wall velocity, $U = \dot{R} = dR/dt$, mass, $m_b$, and internal energy, $E_b$. The airgun variables are the density, $\rho$, the density multiplied by the velocity, $\rho v$, and the internal energy, $e$. \texttt{sol.y} has dimensions of $(3[M \times L)+1] + 4)~\times~N$ where $M$ is the number of grid points per 1 m of source length and L is the source length, so $M\times L+1$ is the total number of spatial grid points. The format of \texttt{sol.y} is shown in Figure~\ref{fig:sol y}.

\begin{figure}[h]
\centering
\includegraphics[width=\textwidth]{Fig_exampleOutputs}
\caption{Example outputs showing (a) bubble radius, (b) acoustic pressure from direct arrival, and (c) space-time plot of pressure inside the source. Source is 1.2 m long with a cross-sectional area of 12.5~in$^2$ at a depth of 7.5~m and pressure of 1000~psi. The port is located at $x=1.2$~m and opens at $t=0$ ms.}
\label{fig:example outputs}
\end{figure}

\newpage
\bibliographystyle{elsarticle-harv}
\bibliography{/Users/lwat054/Documents/Stanford_University/Research/References/Bibliography/library.bib}

\end{document}




